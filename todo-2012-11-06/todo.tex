\documentclass{article}
\title{Seattle Vert Ramp: It's Basically Done!}
\author{William Stein}
\begin{document}
\maketitle

Given what I know, here is my best estimate of what is left to do to
finish the vert ramp.  There are several critical details where we
haven't even decided what we want to do yet (e.g., the endwalls, the
lighting, etc.), and other steps that we simply can't do because we
just don't have the money yet (there is a {\em lot} more that we have
to buy in addition to just the final layer of birch).  My best guess
estimate, if we continue working at a solid pace and don't run out of
money, is that we could finish in February, 2013.  It is possible to
finish sooner if we had a lot more money, organization, and people
were to spend more time; however, it is important to acknowledge the
realities of fatigue, life responsibilities, weather, daylight savings
time, etc., If we do this right, the time we spend skating this ramp
will be {\em much, much} longer than the time we spend building it!
This is a much bigger project than just building a vert ramp -- we are
also building a stadium, and for safety reasons it is critical that we
do everything right, no matter how long it takes.

\begin{enumerate}
\item (done) Finish all the cabling between the first two trusses and true them up.
\item (done) Move the scaffolding tower over in preparation for picking the next truss up.
\item Re-rig the scaffolding for picking up the next middle truss.
\item Pick up the next truss and set it in place (probably Saturday, Nov 10, morning).
\item Put three cross braces between the 2nd and 3rd trusses.
\item Figure out how to hang part of the scaffolding off the side of the ramp safely.
\item Pick up the final end truss and set in place (Saturday, Nov 10).
\item Put in 4 more cross braces between trusses 2 and 3 and 14 more cross braces between 3 and 4 (Sunday, Nov 11).
\item Put in 20+ more cables, and true them all up (Sunday, Nov 11?).
\item Gutters and gravel irrigation system (next week or during the weekend in parallel) -- lots of gravel needs to get carried down the hill (by Sunday, Nov 18).
\item Build walls to set the end walls on top of (or somehow extend the existing metal posts to the the ground). Put the end wall posts in place, somehow (we have to do the end walls now rather than later, because they provide major protection against uplift due to wind blowing under the cover and turning it into a huge kite, and we can't put the top cover on before putting the end wall covers on) -- this will require major carpentry work.  We must decide on what sort of doorway we want on each wall. (Nov 18?)
\item Run wiring for lights.    (before Dec 2!)
\item Install light fixtures.   (before Dec 2!)
\item Clean out areas next to north and south side of the ramp, and raise endwall fabric and tie in place. (Dec 1?)
\item Clean up and lay tarping on the ground along the east side of the ramp, then carry the roof fabric over there, and unroll it.  (Dec 1?)
\item Fill the fabric with pipes as we carefully pull it over the trusses in a coordinated manner (this will involve at least *four* highly skilled climbers...).  (Dec 2?)
\item Tighten the roof fabric down, tensioning a huge number of straps just right (which is pretty involved).  (Dec 2?)
\item String some temporary high-efficiency lights so we can see well enough to work (and skate). (Dec 9)
\item Put in a lot more cross bracing behind the ramp on both sides (this could be done sooner rather than later!). (by Dec 9)
\item Run some huge screws from the points where the trusses tie to
  the pony wall down to the concrete foundations, when possible.
\item After drying the ramp out, assess the situation regarding water and scaffolding damage to the flat bottom.  If it is too damaged, remove all flat bottom 3/4" ply and replace it.  The 3/4" ply on the flat would then be used to sheet end walls.  (by Dec 9)
\item Replace or fix any wood on the transitions that is damaged. 
\item Water seal the bottoms of a bunch of 3/8" ply, letting it dry under the roof a while (maybe we can skip this step, since this ramp should never get wet again).
\item Screw down the third layer of 3/8" ply. (by Dec 16)
\item Find and fix every single point where the 3/8" ply of the third layer isn't ``perfect''. 
\item Screw down the final diagonal layer of birch. 
\item Weld brackets on the steel pipe coping, and weld the coping together, while installing it (Thadd Grossi will order the pipe coping a week before). (January?)
\item Flip all the plywood on both decks over, revealing the good side, and screw it down flush against the coping.  (January?) 
\item Get power to the ramp (possibly cables in a pipe).   (February?)
\item Install circuit box (?)  
\item Install permanent lights. (February?)
\item Remove temporary lights.   (February?)
\item Sheet across the front of the pony wall on each deck.   (January?)
\item Install pool coping on extensions.   (January?)
\item Build a staircase on the southwest corner of the ramp.   (January?)
\item Build a staircase on the southeast corner of the ramp.    (January?)
\item Get an internet connection and setup some always-on webcams.  (February?)
\item Install a donations lock-box;   (February?)
\item Install a killer sound system.  (February?)
\end{enumerate}
\end{document}
